\documentclass[10pt]{article}

\makeatletter

\usepackage{amstext}
\usepackage{amssymb}
\usepackage{stmaryrd}
\DeclareFontFamily{OT1}{cmtex}{}
\DeclareFontShape{OT1}{cmtex}{m}{n}
  {<5><6><7><8>cmtex8
   <9>cmtex9
   <10><10.95><12><14.4><17.28><20.74><24.88>cmtex10}{}
\DeclareFontShape{OT1}{cmtex}{m}{it}
  {<-> ssub * cmtt/m/it}{}
\newcommand{\texfamily}{\fontfamily{cmtex}\selectfont}
\DeclareFontShape{OT1}{cmtt}{bx}{n}
  {<5><6><7><8>cmtt8
   <9>cmbtt9
   <10><10.95><12><14.4><17.28><20.74><24.88>cmbtt10}{}
\DeclareFontShape{OT1}{cmtex}{bx}{n}
  {<-> ssub * cmtt/bx/n}{}
\newcommand{\tex}[1]{\text{\texfamily#1}}	% NEU

\newcommand{\Sp}{\hskip.33334em\relax}
\newcommand{\NB}{\textbf{NB}}
\newcommand{\Todo}[1]{$\langle$\textbf{To do:}~#1$\rangle$}

\makeatother

%\usepackage{haskell}

\usepackage{proof}
\usepackage{fullpage}

\bibliographystyle{plain}

\parskip=\medskipamount
\parindent=0pt

\newcommand{\isa}{\ensuremath{\; {:}{:}{=} \;}}
\newcommand{\ora}{\ensuremath{\;\mid\;}}
\newcommand{\IF}{\ensuremath{\mathtt{ if \;}}}
\newcommand{\THEN}{\ensuremath{\mathtt{\; then \;}}}
\newcommand{\ELSE}{\ensuremath{\mathtt{\; else \;}}}
\newcommand{\TRUE}{\ensuremath{\mathtt{ true \;}}}
\newcommand{\FALSE}{\ensuremath{\mathtt{\; false \;}}}
\newcommand{\BOOL}{\ensuremath{\mathtt{\; Bool \;}}}

\title{Composable Subtyped Lambda Calculus Interpreter}
\author{Perry Alexander \\
  The University of Kansas EECS Department\\
  \texttt{alex@ittc.ku.edu}}

\begin{document}

\maketitle

\section{Introduction}

The objective of Project 3 is to write an interpreter for simply typed
lambda calculus with subtyping and records ($\lambda_{<:}$)
expressions as defined in \emph{Types and Programming
Languages}~\cite{Pie02a}, Chapters 15 and 16.  In addition, you were
to include booleans and the $\IF$ special form.
\section{SubtypedLambdaMonad Module}

\begin{tabbing}\texfamily
~~{\bfseries module}~{\itshape SubtypedLambdaComp}~(~{\itshape Term}(..)~)\\
\texfamily ~~~~~~{\bfseries where}
\end{tabbing}

The \text{\texfamily {\itshape SubtypedLambdaComp}} module provides the basic definitions for
manipulating simply typed lambda calculus expressions with subtyping
and records added using composable interpreters.  As indicated by the
module name, a monad is used to implement the evaluation and type
checking processes.  The data type representing terms and types is
first defined, followed by the type checking function, subtyping
function, and the evaluation function.

\subsection{Data Types}

An abstract type \text{\texfamily \Varid{T}}, is defined to represent the two possible
types in $\lambda_{<:}$.  A constant represents the Boolean
type while a pair of types represents the $\rightarrow$ type former
application.

%  data Retval a = 
%      Value a | Error String
%      deriving (Eq,Show)
%	       
%  instance Monad Retval where
%      Error s >>= k = Error s
%      Value a >>= k = k a 
%      return = Value
%      fail = Error

\begin{tabbing}\texfamily
~~{\bfseries import}~{\itshape {\itshape Control}.{\itshape Monad}.Error}\\
\texfamily ~~{\bfseries import}~{\itshape {\itshape Data}.List}\\
\texfamily \\
\texfamily ~~{\bfseries type}~{\itshape Retval}~a~=~{\itshape ErrorT}~{\itshape String}~{\itshape IO}~a\\
\texfamily \\
\texfamily ~~{\bfseries data}~\Varid{T}~=\\
\texfamily ~~~~~~{\itshape TyBool}~|\\
\texfamily ~~~~~~{\itshape TyTop}~|\\
\texfamily ~~~~~~{\itshape TyArr}~\Varid{T}~\Varid{T}~|\\
\texfamily ~~~~~~{\itshape TyTpl}~[\Varid{T}]~|\\
\texfamily ~~~~~~{\itshape TyRec}~[({\itshape String},\Varid{T})]\\
\texfamily ~~~~~~{\bfseries deriving}~({\itshape Eq},{\itshape Show})\\
\texfamily \\
\texfamily ~~dom~::~\Varid{T}~\char'31~{\itshape Retval}~\Varid{T}\\
\texfamily ~~dom~t~=\\
\texfamily ~~~~~~{\bfseries case}~t~{\bfseries of}\\
\texfamily ~~~~~~~~~~~~~~~~~~~~~~~~~~~~~~~~~~~~~~~~~~~~~~~~~{\itshape TyArr}~d~\char95 ~\char'31~return~d\\
\texfamily ~~~~~~~~~~~~~~~~~~~~~~~~~~~~~~~~~~~~~~~~~~~~~~~~~{\itshape TyBool}~\char'31~fail~\char34 Type~Error~-~Cannot~find~domain~of~a~boolean~type\char34 \\
\texfamily ~~~~~~~~~~~~~~~~~~~~~~~~~~~~~~~~~~~~~~~~~~~~~~~~~{\itshape TyRec}~\char95 ~\char'31~fail~\char34 Type~Error~-~Cannot~find~domain~of~a~record~type~\char34 \\
\texfamily ~~~~~~~~~~~~~{\itshape TyTpl}~\char95 ~\char'31~fail~\char34 Type~Error~-~Cannot~find~the~domain~of~a~tuple~tye\char34 \\
\texfamily ~~~~~~~~~~~~~{\itshape TyTop}~\char'31~fail~\char34 Type~Error~-~Cannot~find~the~domain~of~top\char34 \\
\texfamily \\
\texfamily ~~ran~::~\Varid{T}~\char'31~{\itshape Retval}~\Varid{T}\\
\texfamily ~~ran~t~=\\
\texfamily ~~~~~~{\bfseries case}~t~{\bfseries of}\\
\texfamily ~~~~~~~~~~~~~~~~~~~~~~~~~~~~~~~~~~~~~~~~~~~~~~~~~{\itshape TyArr}~\char95 ~r~\char'31~return~r\\
\texfamily ~~~~~~~~~~~~~~~~~~~~~~~~~~~~~~~~~~~~~~~~~~~~~~~~~{\itshape TyBool}~\char'31~fail~\char34 Type~Error~-~Cannot~find~range~of~a~boolean~type\char34 \\
\texfamily ~~~~~~~~~~~~~{\itshape TyRec}~\char95 ~\char'31~fail~\char34 Type~Error~-~Cannot~find~range~of~a~record~type\char34 \\
\texfamily ~~~~~~~~~~~~~{\itshape TyTpl}~\char95 ~\char'31~fail~\char34 Type~Error~-~Cannot~find~the~range~of~a~tuple~tye\char34 \\
\texfamily ~~~~~~~~~~~~~~~~~~~~~~~~~~~~~~~~~~~~~~~~~~~~~~~~~{\itshape TyTop}~\char'31~fail~\char34 Type~Error~-~Cannot~find~range~of~top\char34 \\
\texfamily \\
\texfamily ~~isArr~::~\Varid{T}~\char'31~{\itshape Bool}\\
\texfamily ~~isArr~({\itshape TyArr}~\char95 ~\char95 )~=~{\itshape True}\\
\texfamily ~~isArr~\char95 ~=~{\itshape False}\\
\texfamily \\
\texfamily ~~isRec~::~\Varid{T}~\char'31~{\itshape Bool}\\
\texfamily ~~isRec~({\itshape TyRec}~\char95 )~=~{\itshape True}\\
\texfamily ~~isRec~\char95 ~=~{\itshape False}\\
\texfamily \\
\texfamily ~~isTpl~::~\Varid{T}~\char'31~{\itshape Bool}\\
\texfamily ~~isTpl~({\itshape TyTpl}~\char95 )~=~{\itshape True}\\
\texfamily ~~isTpl~\char95 ~=~{\itshape False}
\end{tabbing}

A constructed type, \text{\texfamily {\itshape Term}}, is defined to represent the abstract
syntax for $\lambda_{<:}$ terms.  Note that this definition is
identical to that used in $\lambda_{<:}$ except \text{\texfamily {\itshape Lambda}}
includes a type annotation, the \text{\texfamily {\itshape If}} construct is defined, and boolean
costants have been added.  The implementation of the \text{\texfamily {\itshape Term}} type is a
Maybe type that either represents a legal term or a form that cannot
be evaluated.

