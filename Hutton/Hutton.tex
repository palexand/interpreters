\documentclass[10pt]{article}

%% ODER: format ==         = "\mathrel{==}"
%% ODER: format /=         = "\neq "
\makeatletter

\usepackage{amstext}
\usepackage{amssymb}
\usepackage{stmaryrd}
\DeclareFontFamily{OT1}{cmtex}{}
\DeclareFontShape{OT1}{cmtex}{m}{n}
  {<5><6><7><8>cmtex8
   <9>cmtex9
   <10><10.95><12><14.4><17.28><20.74><24.88>cmtex10}{}
\DeclareFontShape{OT1}{cmtex}{m}{it}
  {<-> ssub * cmtt/m/it}{}
\newcommand{\texfamily}{\fontfamily{cmtex}\selectfont}
\DeclareFontShape{OT1}{cmtt}{bx}{n}
  {<5><6><7><8>cmtt8
   <9>cmbtt9
   <10><10.95><12><14.4><17.28><20.74><24.88>cmbtt10}{}
\DeclareFontShape{OT1}{cmtex}{bx}{n}
  {<-> ssub * cmtt/bx/n}{}
\newcommand{\tex}[1]{\text{\texfamily#1}}	% NEU

\newcommand{\Sp}{\hskip.33334em\relax}

\newlength{\lwidth}\setlength{\lwidth}{4.5cm}
\newlength{\cwidth}\setlength{\cwidth}{8mm} % 3mm

\newcommand{\Conid}[1]{\mathit{#1}}
\newcommand{\Varid}[1]{\mathit{#1}}
\newcommand{\anonymous}{\kern0.06em \vbox{\hrule\@width.5em}}
\newcommand{\plus}{\mathbin{+\!\!\!+}}
\newcommand{\bind}{\mathbin{>\!\!\!>\mkern-6.7mu=}}
\newcommand{\sequ}{\mathbin{>\!\!\!>}}
\renewcommand{\leq}{\leqslant}
\renewcommand{\geq}{\geqslant}
\newcommand{\NB}{\textbf{NB}}
\newcommand{\Todo}[1]{$\langle$\textbf{To do:}~#1$\rangle$}

\makeatother

\usepackage{fullpage}
\usepackage{alltt}

\bibliographystyle{plain}

\parskip=\medskipamount
\parindent=0pt

\title{A Haskell Companion for ``Fold and Unfold for Program Semantics''}
\author{\emph{Uk'taad B'mal} \\
  The University of Kansas - ITTC \\
  2335 Irving Hill Rd, Lawrence, KS 66045 \\
  \texttt{lambda@ittc.ku.edu}}

\begin{document}

\maketitle

\begin{abstract}

  This document is a primer to accompany the paper ``Fold and Unfold
  for Program Semantics'' by Graham Hutton.  It attempts to
  re-implement in \texttt{Haskell} the interpreters written in
  \texttt{Gopher}.

\end{abstract}

\section{Introduction}

\section{Fold For Expressions}

Module \ensuremath{\Conid{FoldForExpressions}} presents a Haskell encoding of
Hutton's interpreter and definition of fold over expressions:

\begin{tabbing}
\qquad\=\hspace{\lwidth}\=\hspace{\cwidth}\=\+\kill
${\mathbf{module}\;\Conid{FoldForExpressions}\;\mathbf{where}}$
\end{tabbing}
First, we define data structures for \ensuremath{\Conid{Expr}} representing numbers and
addition.  This is a parameterized version of the expression data type
defined in the Duponcheel interpreters:

\begin{tabbing}
\qquad\=\hspace{\lwidth}\=\hspace{\cwidth}\=\+\kill
${\mathbf{data}\;(\Conid{Show}\;\Varid{a},\Conid{Eq}\;\Varid{a})\Rightarrow \Conid{Expr}\;\Varid{a}}$\\
${\hskip2.00em\relax\mathrel{=}\Conid{Val}\;\Varid{a}}$\\
${\hskip2.00em\relax\mid \Conid{Add}\;(\Conid{Expr}\;\Varid{a})\;(\Conid{Expr}\;\Varid{a})}$\\
${\hskip2.00em\relax\phantom{\mid \mbox{}}\mathbf{deriving}\;(\Conid{Eq},\Conid{Show})}$
\end{tabbing}
It's easy now to define direct evaluation of the expression data
structure using direct recursion.  Note that \ensuremath{\Conid{Expr}} is instantiated
over \ensuremath{\Conid{Int}} in this evaluator:

\begin{tabbing}
\qquad\=\hspace{\lwidth}\=\hspace{\cwidth}\=\+\kill
${\Varid{eval}_1\mathbin{::}\Conid{Expr}\;\Conid{Int}\to \Conid{Int}}$\\
${\Varid{eval}_1\;(\Conid{Val}\;\Varid{n})\mathrel{=}\Varid{n}}$\\
${\Varid{eval}_1\;(\Conid{Add}\;\Varid{x}\;\Varid{y})\mathrel{=}\Varid{eval}_1\;\Varid{x}\mathbin{+}\Varid{eval}_1\;\Varid{y}}$
\end{tabbing}
The function \ensuremath{\Varid{\phi}} is actually defined as \ensuremath{\Varid{deno}} in the Hutton paper.
I have used \ensuremath{\Varid{\phi}} to be consistent with Duponcheel.  The functions
\ensuremath{\Varid{f}} and \ensuremath{\Varid{g}} provide the semantics for \ensuremath{\Conid{Val}} and \ensuremath{\Conid{Add}}.

\begin{tabbing}
\qquad\=\hspace{\lwidth}\=\hspace{\cwidth}\=\+\kill
${\Varid{f}\mathrel{=}\Varid{id}}$\\
${\Varid{g}\mathrel{=}(\mathbin{+})}$\\
${}$\\
${\Varid{\phi}\;(\Conid{Val}\;\Varid{n})\mathrel{=}\Varid{f}\;\Varid{n}}$\\
${\Varid{\phi}\;(\Conid{Add}\;\Varid{x}\;\Varid{y})\mathrel{=}\Varid{g}\;(\Varid{\phi}\;\Varid{x})\;(\Varid{\phi}\;\Varid{y})}$
\end{tabbing}
The \ensuremath{\Varid{fold}} operation defines a general fold function over expressions.
The signature of the fold is interesting.  So much so that I let
\texttt{Haskell}'s type inference system find it for me.  \ensuremath{\Varid{a}} reflects
the type of the value encapsulated by \ensuremath{\Conid{Val}\;\Varid{a}} while \ensuremath{\Varid{b}} reflects the
domain of what will be the evaluation function.  Specifically, \ensuremath{\Varid{a}\to \Varid{b}}
defines the signature of the value extraction function.  For \ensuremath{\Varid{eval}_2},
that function is \ensuremath{\Varid{id}} because we are simply extracting the value.  For
\ensuremath{\Varid{comp}} later, it will be \ensuremath{\Conid{Inst}}, the instruction generated for the
stack machine.

\begin{tabbing}
\qquad\=\hspace{\lwidth}\=\hspace{\cwidth}\=\+\kill
${\Varid{fold}\mathbin{::}(\Conid{Eq}\;\Varid{a},\Conid{Show}\;\Varid{a})\Rightarrow (\Varid{a}\to \Varid{b})\to (\Varid{b}\to \Varid{b}\to \Varid{b})\to \Conid{Expr}\;\Varid{a}\to \Varid{b}}$\\
${\Varid{fold}\;\Varid{f}\;\Varid{g}\;(\Conid{Val}\;\Varid{n})\mathrel{=}\Varid{f}\;\Varid{n}}$\\
${\Varid{fold}\;\Varid{f}\;\Varid{g}\;(\Conid{Add}\;\Varid{x}\;\Varid{y})\mathrel{=}\Varid{g}\;(\Varid{fold}\;\Varid{f}\;\Varid{g}\;\Varid{x})\;(\Varid{fold}\;\Varid{f}\;\Varid{g}\;\Varid{y})}$
\end{tabbing}
The evaluation function, \ensuremath{\Varid{eval}_1}, defined earlier can now be redefined
using \ensuremath{\Varid{fold}} and specifying the semantic mappings for \ensuremath{\Conid{Val}} and \ensuremath{\Conid{Add}}
respectively.

\begin{tabbing}
\qquad\=\hspace{\lwidth}\=\hspace{\cwidth}\=\+\kill
${\Varid{eval}_2\mathrel{=}\Varid{fold}\;\Varid{id}\;(\mathbin{+})}$
\end{tabbing}
Hutton also defines a compiler function, \ensuremath{\Varid{comp}}, that generates a
sequence of instructions for a stack machine.  \ensuremath{\Conid{Inst}} is a data type
representing possible instructions for the machine.  

\begin{tabbing}
\qquad\=\hspace{\lwidth}\=\hspace{\cwidth}\=\+\kill
${\mathbf{data}\;\Conid{Inst}\mathrel{=}\Conid{PUSH}\;\Conid{Int}\mid \Conid{ADD}\;\mathbf{deriving}\;(\Conid{Eq},\Conid{Show})}$
\end{tabbing}
\ensuremath{\Varid{comp}} translates an expression defined over integers into a list of
instructions.  When looking at the \ensuremath{\Varid{fold}}, \ensuremath{\Varid{a}} is instantiated with
\ensuremath{\Conid{Expr}\;\Conid{Int}} while \ensuremath{\Varid{b}} is instantiated with \ensuremath{[\mskip1.5mu \Conid{Inst}\mskip1.5mu]}.  Initially, I was
confused with the definition of \ensuremath{\Varid{g}}.  The fold will actually evaluate
the arguments to \ensuremath{\Conid{Add}} and return them.  It thus makes sense that \ensuremath{\Varid{g}}
simply concatenates the lists of instructions and tacks an \ensuremath{\Conid{ADD}} onto
the end.

\begin{tabbing}
\qquad\=\hspace{\lwidth}\=\hspace{\cwidth}\=\+\kill
${\Varid{comp}\mathbin{::}\Conid{Expr}\;\Conid{Int}\to [\mskip1.5mu \Conid{Inst}\mskip1.5mu]}$\\
${\Varid{comp}\mathrel{=}\Varid{fold}\;\Varid{f}\;\Varid{g}}$\\
${\phantom{\Varid{comp}\mathrel{=}\mbox{}}\mathbf{where}}$\\
${\phantom{\Varid{comp}\mathrel{=}\mbox{}}\hskip1.00em\relax\Varid{f}\;\Varid{n}\mathrel{=}[\mskip1.5mu \Conid{PUSH}\;\Varid{n}\mskip1.5mu]}$\\
${\phantom{\Varid{comp}\mathrel{=}\mbox{}}\hskip1.00em\relax\Varid{g}\;\Varid{xs}\;\Varid{ys}\mathrel{=}\Varid{xs}\plus \Varid{ys}\plus [\mskip1.5mu \Conid{ADD}\mskip1.5mu]}$
\end{tabbing}
Hutton extends \ensuremath{\Varid{eval}_2} to include variables and a program store.  I'm
going to extend \ensuremath{\Varid{eval}_2} like I did for Duponcheel by adding a multiply
operation.  It's not particularly difficult, however it becomes clear
that the interpreter is not particularly modular or extensible.

First, we need to extend the expression data structure to include the multiplication operation:

\begin{tabbing}
\qquad\=\hspace{\lwidth}\=\hspace{\cwidth}\=\+\kill
${\mathbf{data}\;(\Conid{Show}\;\Varid{a},\Conid{Eq}\;\Varid{a})\Rightarrow \Varid{Expr}_1\;\Varid{a}}$\\
${\hskip2.00em\relax\mathrel{=}\Conid{Val1}\;\Varid{a}}$\\
${\hskip2.00em\relax\mid \Conid{Add1}\;(\Varid{Expr}_1\;\Varid{a})\;(\Varid{Expr}_1\;\Varid{a})}$\\
${\hskip2.00em\relax\mid \Conid{Mul1}\;(\Varid{Expr}_1\;\Varid{a})\;(\Varid{Expr}_1\;\Varid{a})}$\\
${\hskip2.00em\relax\phantom{\mid \mbox{}}\mathbf{deriving}\;(\Conid{Eq},\Conid{Show})}$
\end{tabbing}
Then we have to redefine \ensuremath{\Varid{fold}} as \ensuremath{\Varid{fold1}} to take three functions to
include the semantics for multiply:

\begin{tabbing}
\qquad\=\hspace{\lwidth}\=\hspace{\cwidth}\=\+\kill
${\Varid{fold1}\mathbin{::}(\Conid{Show}\;\Varid{a},\Conid{Eq}\;\Varid{a})\Rightarrow (\Varid{a}\to \Varid{b})\to (\Varid{b}\to \Varid{b}\to \Varid{b})\to (\Varid{b}\to \Varid{b}\to \Varid{b})\to \Varid{Expr}_1\;\Varid{a}\to \Varid{b}}$\\
${\Varid{fold1}\;\Varid{f}\;\Varid{g}\;\Varid{h}\;(\Conid{Val1}\;\Varid{x})\mathrel{=}\Varid{f}\;\Varid{x}}$\\
${\Varid{fold1}\;\Varid{f}\;\Varid{g}\;\Varid{h}\;(\Conid{Add1}\;\Varid{x}\;\Varid{y})\mathrel{=}\Varid{g}\;(\Varid{fold1}\;\Varid{f}\;\Varid{g}\;\Varid{h}\;\Varid{x})\;(\Varid{fold1}\;\Varid{f}\;\Varid{g}\;\Varid{h}\;\Varid{y})}$\\
${\Varid{fold1}\;\Varid{f}\;\Varid{g}\;\Varid{h}\;(\Conid{Mul1}\;\Varid{x}\;\Varid{y})\mathrel{=}\Varid{h}\;(\Varid{fold1}\;\Varid{f}\;\Varid{g}\;\Varid{h}\;\Varid{x})\;(\Varid{fold1}\;\Varid{f}\;\Varid{g}\;\Varid{h}\;\Varid{y})}$
\end{tabbing}
Now we can define \ensuremath{\Varid{eval}_3} to perform the appropriate evaluation:

\begin{tabbing}
\qquad\=\hspace{\lwidth}\=\hspace{\cwidth}\=\+\kill
${\Varid{eval}_3\mathrel{=}\Varid{fold1}\;\Varid{id}\;(\mathbin{+})\;(\mathbin{*})}$
\end{tabbing}
Hutton continues to generize the evaluation function by adding a
\ensuremath{\Conid{Store}} and variable access to the interpetation.  He does not provide
a full definition for \ensuremath{\Conid{Store}}, so we'll try to here.  First we define
a new \ensuremath{\Conid{Expr}} type that includes the concept of a variable:

\begin{tabbing}
\qquad\=\hspace{\lwidth}\=\hspace{\cwidth}\=\+\kill
${\mathbf{data}\;\Varid{Expr}_2\;\Varid{ty}}$\\
${\hskip2.00em\relax\mathrel{=}\Conid{Val2}\;\Varid{ty}}$\\
${\hskip2.00em\relax\mid \Conid{Add2}\;(\Varid{Expr}_2\;\Varid{ty})\;(\Varid{Expr}_2\;\Varid{ty})}$\\
${\hskip2.00em\relax\mid \Conid{Var2}\;\Conid{String}}$
\end{tabbing}
Note that the \ensuremath{\Varid{Expr}_2} type is parameterized over the contained data,
\ensuremath{\Varid{n}}, and the variable name type, \ensuremath{\Varid{v}}.  This will allow the most
general possible \ensuremath{\Conid{Store}} to be defined.

Next, we define a new \ensuremath{\Varid{fold}} that includes the processing of variable
references.  \ensuremath{\Varid{h}} now provides a lookup capability for variables in the
store.

\begin{tabbing}
\qquad\=\hspace{\lwidth}\=\hspace{\cwidth}\=\+\kill
${\Varid{fold2}\;\Varid{f}\;\Varid{g}\;\Varid{h}\;(\Conid{Val2}\;\Varid{x})\mathrel{=}\Varid{f}\;\Varid{x}}$\\
${\Varid{fold2}\;\Varid{f}\;\Varid{g}\;\Varid{h}\;(\Conid{Add2}\;\Varid{x}\;\Varid{y})\mathrel{=}\Varid{g}\;(\Varid{fold2}\;\Varid{f}\;\Varid{g}\;\Varid{h}\;\Varid{x})\;(\Varid{fold2}\;\Varid{f}\;\Varid{g}\;\Varid{h}\;\Varid{y})}$\\
${\Varid{fold2}\;\Varid{f}\;\Varid{g}\;\Varid{h}\;(\Conid{Var2}\;\Varid{c})\mathrel{=}\Varid{h}\;\Varid{c}}$
\end{tabbing}
We still need a \ensuremath{\Conid{Store}} and \ensuremath{\Varid{find}} function for records in the store.
The easiest \ensuremath{\Conid{Store}} to define is a list of pairs whose first element
is a \ensuremath{\Conid{String}} and whose second is a value:

\begin{tabbing}
\qquad\=\hspace{\lwidth}\=\hspace{\cwidth}\=\+\kill
${\mathbf{data}\;\Conid{Binding}\;\Varid{n}\mathrel{=}\Conid{Binding}\;\Conid{String}\;\Varid{n}}$\\
${}$\\
${\mathbf{type}\;\Conid{Store}\mathrel{=}[\mskip1.5mu (\Conid{Binding}\;\Conid{Int})\mskip1.5mu]}$\\
${}$\\
${\Varid{find}\mathbin{::}\Conid{String}\to \Conid{Store}\to \Conid{Int}}$\\
${\Varid{find}\;\Varid{v}\;[\mskip1.5mu \mskip1.5mu]\mathrel{=}\Varid{error}\;(\text{\tt \char34 Variable~\char34}\plus \Varid{v}\plus \text{\tt \char34 ~not~found~in~store\char34})}$\\
${\Varid{find}\;\Varid{n1}\;((\Conid{Binding}\;\Varid{n2}\;\Varid{z})\mathbin{:}\Varid{bs})\mathrel{=}\mathbf{if}\;(\Varid{n1}\equiv \Varid{n2})\;\mathbf{then}\;\Varid{z}\;\mathbf{else}\;\Varid{find}\;\Varid{n1}\;\Varid{bs}}$
\end{tabbing}
Now we can write the \ensuremath{\Varid{eval}} function over this new data structure with
an associated store like Hutton:

\begin{tabbing}
\qquad\=\hspace{\lwidth}\=\hspace{\cwidth}\=\+\kill
${\Varid{evalS}\mathbin{::}(\Varid{Expr}_2\;\Conid{Int})\to (\Conid{Store}\to \Conid{Int})}$\\
${\Varid{evalS}\mathrel{=}\Varid{fold2}\;\Varid{f}\;\Varid{g}\;\Varid{h}}$\\
${\phantom{\Varid{evalS}\mathrel{=}\mbox{}}\mathbf{where}}$\\
${\phantom{\Varid{evalS}\mathrel{=}\mbox{}}\hskip1.50em\relax\Varid{f}\;\Varid{n}\mathrel{=}\lambda \Varid{s}\to \Varid{n}}$\\
${\phantom{\Varid{evalS}\mathrel{=}\mbox{}}\hskip1.50em\relax\Varid{g}\;\Varid{fx}\;\Varid{fy}\mathrel{=}\lambda \Varid{s}\to \Varid{fx}\;\Varid{s}\mathbin{+}\Varid{fy}\;\Varid{s}}$\\
${\phantom{\Varid{evalS}\mathrel{=}\mbox{}}\hskip1.50em\relax\Varid{h}\;\Varid{n}\mathrel{=}\lambda \Varid{s}\to \Varid{find}\;\Varid{n}\;\Varid{s}}$
\end{tabbing}
This function maps an expression to a function mapping \ensuremath{\Conid{Store}} to
\ensuremath{\Conid{Int}} rather than just an \ensuremath{\Conid{Int}}.  We must in effect provide a \ensuremath{\Conid{Store}}
to perform a full evaluation.  However, that \ensuremath{\Conid{Store}} is not mutable -
there is no mechanism for adding to or deleting from the \ensuremath{\Conid{Store}}.

Some interesting examples:

\begin{tabbing}
\qquad\=\hspace{\lwidth}\=\hspace{\cwidth}\=\+\kill
${\Varid{testS}\mathrel{=}(\Conid{Add2}\;(\Conid{Var2}\;\text{\tt \char34 Test\char34})\;(\Conid{Var2}\;\text{\tt \char34 x\char34}))}$\\
${\Varid{failS}\mathrel{=}(\Conid{Add2}\;(\Conid{Var2}\;\text{\tt \char34 test\char34})\;(\Conid{Var2}\;\text{\tt \char34 x\char34}))}$\\
${\Varid{storeS}\mathbin{::}\Conid{Store}\mathrel{=}[\mskip1.5mu (\Conid{Binding}\;\text{\tt \char34 Test\char34}\;\mathrm{3}),(\Conid{Binding}\;\text{\tt \char34 x\char34}\;\mathrm{4})\mskip1.5mu]}$
\end{tabbing}
You can now enter the following at the command line to see what's going on:

\begin{tabbing}
\qquad\=\hspace{\lwidth}\=\hspace{\cwidth}\=\+\kill
${\Varid{evalS}\;\Varid{storeS}\;\Varid{testS}}$\\
${}$\\
${\Varid{evalS}\;\Varid{storeS}\;\Varid{failS}}$
\end{tabbing}
This is interesting from a semantics definition perspective.  However,
it is not a particularly useful way to define modular interpreters as
Duponcheel does.

One question is whether the \ensuremath{\Conid{Sum}} type used buy Duponcheel to form
composite data types and composite application functions.  Hutton's
approach uses the position of the folded function in the argument list
to determine which function to apply.  Duponcheel defines a function
composition operation over the \ensuremath{\Conid{Sum}} type to accomplish a similar task.

First, define new data types for each element of the expression:

\begin{tabbing}
\qquad\=\hspace{\lwidth}\=\hspace{\cwidth}\=\+\kill
${\mathbf{data}\;\Varid{Expr}_2\;\Varid{a}\mathrel{=}\Conid{Val2}\;\Varid{a}}$\\
${\mathbf{data}\;\Varid{Expr}_3\;\Varid{a}\mathrel{=}\Conid{Add2}\;\Varid{a}\;\Varid{a}}$
\end{tabbing}
Now reproduce the \ensuremath{\Conid{Sum}} definition from Duponcheel:

\begin{tabbing}
\qquad\=\hspace{\lwidth}\=\hspace{\cwidth}\=\+\kill
${\mathbf{data}\;\Conid{Sum}\;\Varid{a}\;\Varid{b}}$\\
${\hskip2.00em\relax\mathrel{=}\Conid{L}\;\Varid{a}}$\\
${\hskip2.00em\relax\phantom{\mathrel{=}\mbox{}}\Conid{R}\;\Varid{b}}$\\
${}$\\
${(\mathbin{<+>})\mathbin{::}(\Varid{x}\to \Varid{z})\to (\Varid{y}\to \Varid{z})\to ((\Conid{Sum}\;\Varid{x}\;\Varid{y})\to \Varid{z})}$\\
${\Varid{f}\mathbin{<+>}\Varid{g}\mathrel{=}\Varid{s}\to \mathbf{case}\;\Varid{s}\;\mathbf{of}}$\\
${\hskip11.50em\relax(\Conid{L}\;\Varid{x})\to \Varid{f}\;\Varid{x}}$\\
${\hskip11.50em\relax(\Conid{R}\;\Varid{x})\to \Varid{g}\;\Varid{x}}$
\end{tabbing}
Now define a \ensuremath{\Conid{Term}} type to represent both possible term elements:

\begin{tabbing}
\qquad\=\hspace{\lwidth}\=\hspace{\cwidth}\=\+\kill
${\mathbf{type}\;\Conid{Term}\mathrel{=}\Conid{Sum}\;(\Varid{Expr}_2\;\Conid{Int})\;(\Varid{Expr}_3\;\Conid{Int})}$
\end{tabbing}
Finally, define a mapping function from the composite of the \ensuremath{\Varid{id}} and
\ensuremath{(\mathbin{+})} operations:

\begin{tabbing}
\qquad\=\hspace{\lwidth}\=\hspace{\cwidth}\=\+\kill
${\Varid{j}\mathrel{=}\Varid{id}\mathbin{<+>}(\mathbin{+})}$
\end{tabbing}
If all goes well, we should be able to fold \ensuremath{\Varid{j}} through the structure.

\begin{tabbing}
\qquad\=\hspace{\lwidth}\=\hspace{\cwidth}\=\+\kill
${\Varid{fold2}\;\Varid{j}\;(\Conid{L}\;(\Conid{Val2}\;\Varid{x}))\mathrel{=}\Varid{j}\;\Varid{x}}$\\
${\Varid{fold2}\;\Varid{j}\;(\Conid{R}\;(\Conid{Add2}\;\Varid{x}\;\Varid{y}))\mathrel{=}\Varid{j}\;(\Varid{fold2}\;\Varid{j}\;\Varid{x})\;(\Varid{fold2}\;\Varid{j}\;\Varid{y})}$
\end{tabbing}
But all doesn't go well.  The problem is that the arity of \ensuremath{\Varid{j}} is
fixed.  Thus, there is no way to apply it to both \ensuremath{\Varid{x}} in the first
case and the results of \ensuremath{\Varid{fold2}} applied twice in the second case.
What's happening here is actually pretty simple.  Hutton's fold
semantics unpackages the arguments to \ensuremath{\Conid{Add2}} before applying \ensuremath{\Conid{Fold2}}.
This explains why: (i) the signature for \ensuremath{\Varid{g}} above maps two elements
of the type \emph{encapsulated by} \ensuremath{\Conid{Val}} to an element of the same
type rather than two elements of the \ensuremath{\Conid{Expr}}; and (ii) why the function
composition cannot be applied in the same way.  It would be
interesting to attempt to rewrite the fold to achieve this end.
\section{Unfold For Expressions}

Module \ensuremath{\Conid{UnfoldForExpressions}} presents a Haskell encoding of
Hutton's interpreter and definition of fold over expressions:

\begin{tabbing}
\qquad\=\hspace{\lwidth}\=\hspace{\cwidth}\=\+\kill
${\mathbf{module}\;\Conid{UnfoldForExpressions}\;\mathbf{where}}$
\end{tabbing}
\begin{tabbing}
\qquad\=\hspace{\lwidth}\=\hspace{\cwidth}\=\+\kill
${\mathbf{data}\;(\Conid{Show}\;\Varid{a},\Conid{Eq}\;\Varid{a})\Rightarrow \Conid{Expr}\;\Varid{a}}$\\
${\hskip2.00em\relax\mathrel{=}\Conid{Val}\;\Varid{a}}$\\
${\hskip2.00em\relax\mid \Conid{Add}\;(\Conid{Expr}\;\Varid{a})\;(\Conid{Expr}\;\Varid{a})}$\\
${\hskip2.00em\relax\phantom{\mid \mbox{}}\mathbf{deriving}\;(\Conid{Eq},\Conid{Show})}$
\end{tabbing}
The translation function is interesting as it almost directly
implements the operational semantics rules defined for expressions and
exhibits why it is an unfold.  What \ensuremath{\Varid{trans}} does is takes an
expression and generates a list of possible transformations of that
expression.  Because the semantics of \ensuremath{\Conid{Add}} don't specify whether the
first or second argument is evaluated first, there are two possible
paths.  Thus, \ensuremath{\Varid{trans}} is effectively a one-step unfold.

\begin{tabbing}
\qquad\=\hspace{\lwidth}\=\hspace{\cwidth}\=\+\kill
${\Varid{trans}\mathbin{::}\Conid{Expr}\;\Conid{Int}\to [\mskip1.5mu \Conid{Expr}\;\Conid{Int}\mskip1.5mu]}$\\
${\Varid{trans}\;(\Conid{Val}\;\Varid{n})\mathrel{=}[\mskip1.5mu \mskip1.5mu]}$\\
${\Varid{trans}\;(\Conid{Add}\;(\Conid{Val}\;\Varid{n})\;(\Conid{Val}\;\Varid{m}))\mathrel{=}[\mskip1.5mu \Conid{Val}\;(\Varid{n}\mathbin{+}\Varid{m})\mskip1.5mu]}$\\
${\Varid{trans}\;(\Conid{Add}\;\Varid{x}\;\Varid{y})}$\\
${\phantom{\Varid{trans}\;\mbox{}}\mathrel{=}[\mskip1.5mu \Conid{Add}\;\Varid{x'}\;\Varid{y}\mid \Varid{x'}\leftarrow \Varid{trans}\;\Varid{x}\mskip1.5mu]\plus }$\\
${\phantom{\Varid{trans}\;\mbox{}}\phantom{\mathrel{=}\mbox{}}[\mskip1.5mu \Conid{Add}\;\Varid{x}\;\Varid{y'}\mid \Varid{y'}\leftarrow \Varid{trans}\;\Varid{y}\mskip1.5mu]}$
\end{tabbing}
Hutton now defines \ensuremath{\Varid{exec}} as a function that applies \ensuremath{\Varid{trans}}
repeatedly until nothing remains to be translated.  The result is a
tree whose nodes are the results of applying \ensuremath{\Varid{trans}} to their
immediate predecessor in the tree.  So, each list generated by \ensuremath{\Varid{trans}}
is treated as the data associated with a set of sub-nodes.  To
represent this, we'll generate a tree.  First, define a data type for
a tree with arbitrary subtrees:

\begin{tabbing}
\qquad\=\hspace{\lwidth}\=\hspace{\cwidth}\=\+\kill
${\mathbf{data}\;\Conid{Tree}\;\Varid{a}\mathrel{=}\Conid{Node}\;\Varid{a}\;[\mskip1.5mu \Conid{Tree}\;\Varid{a}\mskip1.5mu]\;\mathbf{deriving}\;\Conid{Show}}$
\end{tabbing}
Now \ensuremath{\Varid{exec}} can be defined recursively using list comprehension.
Executing \ensuremath{\Varid{e}} is \ensuremath{\Varid{e}} itself in a node with the possible translations
of \ensuremath{\Varid{e}}.  Because \ensuremath{\Varid{exec}} is referenced recursively in the list
comprehension, it will be applied to the results of each application
of \ensuremath{\Varid{trans}}.  This is a very powerful and interesting application of
\texttt{Haskell}'s list comprehension operator!

\begin{tabbing}
\qquad\=\hspace{\lwidth}\=\hspace{\cwidth}\=\+\kill
${\Varid{exec}\mathbin{::}\Conid{Expr}\;\Conid{Int}\to \Conid{Tree}\;(\Conid{Expr}\;\Conid{Int})}$\\
${\Varid{exec}\;\Varid{e}\mathrel{=}\Conid{Node}\;\Varid{e}\;[\mskip1.5mu \Varid{exec}\;\Varid{e'}\mid \Varid{e'}\leftarrow \Varid{trans}\;\Varid{e}\mskip1.5mu]}$
\end{tabbing}
Hutton abstracts from the case for \ensuremath{\Varid{exec}} by defining a function,
\ensuremath{\Varid{oper}}, that defines the operational semantics for a tree structure
using arbitrary functions for \ensuremath{\Varid{id}} and \ensuremath{\Varid{trans}} as they appear in
\ensuremath{\Varid{exec}}.  Note that \ensuremath{\Varid{id}} does not explicitly appear, but is implicitly
called on the input argument.  \ensuremath{\Varid{oper}} as defined in Hutton will not
typecheck because \ensuremath{\Varid{f}} and \ensuremath{\Varid{g}} are free.  Here is a definition for
\ensuremath{\Varid{oper}} that will type check and function appropriately:

\begin{tabbing}
\qquad\=\hspace{\lwidth}\=\hspace{\cwidth}\=\+\kill
${\Varid{oper}\;\Varid{f}\;\Varid{g}\;\Varid{x}\mathrel{=}\Conid{Node}\;(\Varid{f}\;\Varid{x})\;[\mskip1.5mu \Varid{oper}\;\Varid{f}\;\Varid{g}\;\Varid{x'}\mid \Varid{x'}\leftarrow \Varid{g}\;\Varid{x}\mskip1.5mu]}$
\end{tabbing}
If we replace \ensuremath{\Varid{oper}} with \ensuremath{\Varid{unfold}}, we get a general unfold operation for
trees of any type:

\begin{tabbing}
\qquad\=\hspace{\lwidth}\=\hspace{\cwidth}\=\+\kill
${\Varid{unfold}\;\Varid{f}\;\Varid{g}\;\Varid{x}\mathrel{=}\Conid{Node}\;(\Varid{f}\;\Varid{x})\;[\mskip1.5mu \Varid{unfold}\;\Varid{f}\;\Varid{g}\;\Varid{x'}\mid \Varid{x'}\leftarrow \Varid{g}\;\Varid{x}\mskip1.5mu]}$
\end{tabbing}
and we an redefine \ensuremath{\Varid{oper}} as \ensuremath{\Varid{oper1}}:

\begin{tabbing}
\qquad\=\hspace{\lwidth}\=\hspace{\cwidth}\=\+\kill
${\Varid{oper1}\mathrel{=}\Varid{unfold}\;\Varid{id}\;\Varid{trans}}$
\end{tabbing}
Following are some test cases for \ensuremath{\Varid{trans}}, \ensuremath{\Varid{eval}}, \ensuremath{\Varid{oper}} and \ensuremath{\Varid{oper1}}

\begin{tabbing}
\qquad\=\hspace{\lwidth}\=\hspace{\cwidth}\=\+\kill
${\Varid{test}_0\mathbin{::}\Conid{Expr}\;\Conid{Int}}$\\
${\Varid{test}_0\mathrel{=}(\Conid{Add}\;(\Conid{Val}\;\mathrm{1})\;(\Conid{Val}\;\mathrm{2}))}$\\
${\Varid{test}_1\mathbin{::}\Conid{Expr}\;\Conid{Int}}$\\
${\Varid{test}_1\mathrel{=}(\Conid{Add}\;(\Conid{Add}\;(\Conid{Val}\;\mathrm{1})\;(\Conid{Val}\;\mathrm{2}))\;(\Conid{Val}\;\mathrm{3}))}$\\
${\Varid{test}_2\mathbin{::}\Conid{Expr}\;\Conid{Int}}$\\
${\Varid{test}_2\mathrel{=}(\Conid{Add}\;(\Conid{Add}\;(\Conid{Val}\;\mathrm{1})\;(\Conid{Val}\;\mathrm{2}))\;(\Conid{Add}\;(\Conid{Val}\;\mathrm{3})\;(\Conid{Val}\;\mathrm{4})))}$
\end{tabbing}
\section{Functors, Algebras and Catamorphsisms}

\begin{tabbing}
\qquad\=\hspace{\lwidth}\=\hspace{\cwidth}\=\+\kill
${\mathbf{module}\;\Conid{Cata}\;\mathbf{where}}$
\end{tabbing}
The catamorphism and fold depend on the definition of the standard
datatype least fixpoint.  \ensuremath{\Conid{Fix}} type defines a template for fixed
point data tyeps.  Note that the \ensuremath{\Conid{In}} constructor is required by
\texttt{Haskell} to create instances of \ensuremath{\Conid{Fix}}.  The \ensuremath{\Varid{out}} function is
effectively the opposite of the \ensuremath{\Conid{In}} constructor and pulls the
encapsulated data structure out of the constructor.

\begin{tabbing}
\qquad\=\hspace{\lwidth}\=\hspace{\cwidth}\=\+\kill
${\mathbf{newtype}\;\Conid{Fix}\;\Varid{f}\mathrel{=}\Conid{In}\;(\Varid{f}\;(\Conid{Fix}\;\Varid{f}))}$\\
${}$\\
${\Varid{out}\mathbin{::}\Conid{Fix}\;\Varid{f}\to \Varid{f}\;(\Conid{Fix}\;\Varid{f})}$\\
${\Varid{out}\;(\Conid{In}\;\Varid{x})\mathrel{=}\Varid{x}}$
\end{tabbing}
Now we add the Algebra and Co-Algebraic notation. The original type
for fold:

\begin{tabbing}
\qquad\=\hspace{\lwidth}\=\hspace{\cwidth}\=\+\kill
${\Varid{fold}\mathbin{::}(\Conid{Functor}\;\Varid{f})\Rightarrow (\Varid{f}\;\Varid{a}\to \Varid{a})\to \Conid{Fix}\;\Varid{f}\to \Varid{a}}$
\end{tabbing}
becomes:

\begin{tabbing}
\qquad\=\hspace{\lwidth}\=\hspace{\cwidth}\=\+\kill
${\Varid{fold}\mathbin{::}(\Conid{Functor}\;\Varid{f})\Rightarrow (\Conid{Algebra}\;\Varid{f}\;\Varid{a})\to \Conid{Fix}\;\Varid{f}\to \Varid{a}}$
\end{tabbing}
using the type classes \ensuremath{\Conid{Functor}} and \ensuremath{\Conid{Algebra}}.  Even simpler, we can
use the (Co)AlgebraConstructor class:

\begin{tabbing}
\qquad\=\hspace{\lwidth}\=\hspace{\cwidth}\=\+\kill
${\Varid{pCata}\mathbin{::}(\Conid{AlgebraConstructor}\;\Varid{f}\;\Varid{a})\Rightarrow \Conid{Fix}\;\Varid{f}\to \Varid{a}}$\\
${\Varid{pCata}\mathrel{=}\Varid{\phi}\mathbin{\circ}\Varid{fmap}\;\Varid{pCata}\mathbin{\circ}\Varid{out}}$
\end{tabbing}
Informally, we can define \ensuremath{\Varid{pCata}} as:

\begin{tabbing}
\qquad\=\hspace{\lwidth}\=\hspace{\cwidth}\=\+\kill
${\Varid{pCata}\equiv \Varid{polytypic}\;\Conid{Catamorphism}}$
\end{tabbing}
Note the strong similarity between the polymorphic Catamorphism and
the polymorphic fold, just above.  \ensuremath{\Varid{fold}} \emph{asks for} function
\ensuremath{\Varid{\phi}\;(\Conid{F}\;\Varid{a}\to \Varid{a})} as an argument while \ensuremath{\Varid{pCata}} \emph{knows} \ensuremath{\Varid{\phi}}.

We now take these concepts and encode them as \texttt{Haskell} type
classes and types.  First, define \ensuremath{\Conid{Algebra}}, \ensuremath{\Conid{CoAlgebra}},
\ensuremath{\Conid{AlgebraConstructor}} and \ensuremath{\Conid{CoAlgebraConstructor}}:

\begin{tabbing}
\qquad\=\hspace{\lwidth}\=\hspace{\cwidth}\=\+\kill
${\mathbf{type}\;\Conid{Algebra}\;\Varid{f}\;\Varid{a}\mathrel{=}\Varid{f}\;\Varid{a}\to \Varid{a}}$\\
${\mathbf{type}\;\Conid{CoAlgebra}\;\Varid{f}\;\Varid{a}\mathrel{=}\Varid{a}\to \Varid{f}\;\Varid{a}}$\\
${}$\\
${\mathbf{class}\;\Conid{Functor}\;\Varid{f}\Rightarrow \Conid{AlgebraConstructor}\;\Varid{f}\;\Varid{a}}$\\
${\hskip2.00em\relax\mathbf{where}\;\Varid{\phi}\mathbin{::}\Conid{Algebra}\;\Varid{f}\;\Varid{a}}$\\
${}$\\
${\mathbf{class}\;\Conid{Functor}\;\Varid{f}\Rightarrow \Conid{CoAlgebraConstructor}\;\Varid{f}\;\Varid{a}}$\\
${\hskip2.00em\relax\mathbf{where}\;\Varid{psi}\mathbin{::}\Conid{CoAlgebra}\;\Varid{f}\;\Varid{a}}$
\end{tabbing}
We now redefine \ensuremath{\Varid{fold}} using \ensuremath{\Conid{Functor}} and \ensuremath{\Conid{Algebra}}.  This is very
similar to Duponcheel.

\begin{tabbing}
\qquad\=\hspace{\lwidth}\=\hspace{\cwidth}\=\+\kill
${\mbox{\qquad-{}-  fold :: (Functor f) => (f a -> a) -> Fix f -> a}}$\\
${\Varid{fold}\mathbin{::}(\Conid{Functor}\;\Varid{f})\Rightarrow (\Conid{Algebra}\;\Varid{f}\;\Varid{a})\to \Conid{Fix}\;\Varid{f}\to \Varid{a}}$\\
${\Varid{fold}\;\Varid{g}\mathrel{=}\Varid{g}\mathbin{\circ}\Varid{fmap}\;(\Varid{fold}\;\Varid{g})\mathbin{\circ}\Varid{out}}$
\end{tabbing}
We can similarly define \ensuremath{\Varid{pCata}} using the \ensuremath{\Conid{AlgebraConstructor}} type class:

\begin{tabbing}
\qquad\=\hspace{\lwidth}\=\hspace{\cwidth}\=\+\kill
${\Varid{pCata}\mathbin{::}(\Conid{AlgebraConstructor}\;\Varid{f}\;\Varid{a})\Rightarrow \Conid{Fix}\;\Varid{f}\to \Varid{a}}$\\
${\Varid{pCata}\mathrel{=}\Varid{\phi}\mathbin{\circ}\Varid{fmap}\;\Varid{pCata}\mathbin{\circ}\Varid{out}}$
\end{tabbing}
Finally, we re-define the denotational semantics for the simple
expression language that we began with in \texttt{FoldUnfold.lhs}
using, \ensuremath{\Conid{Algebra}}, datatype fixpoint and polytypic fold.  Note that it
is a little more complex because we carefully maintained generality
over the specific numeric type used for constant values. The type,
\ensuremath{\Conid{E}}, requires two type arguments, and everywhere we have to add the
constraint \ensuremath{(\Conid{Num}\;\Varid{a})}.

\begin{tabbing}
\qquad\=\hspace{\lwidth}\=\hspace{\cwidth}\=\+\kill
${\mathbf{data}\;(\Conid{Num}\;\Varid{numType})\Rightarrow \Conid{E}\;\Varid{numType}\;\Varid{e}\mathrel{=}\Conid{Val}\;\Varid{numType}}$\\
${\phantom{\mathbf{data}\;(\Conid{Num}\;\Varid{numType})\Rightarrow \Conid{E}\;\Varid{numType}\;\Varid{e}\mbox{}}\mid \Conid{Add}\;\Varid{e}\;\Varid{e}}$\\
${}$\\
${\mathbf{type}\;\Conid{Expr}\;\Varid{numType}\mathrel{=}\Conid{Fix}\;(\Conid{E}\;\Varid{numType})}$\\
${}$\\
${\mathbf{instance}\;(\Conid{Num}\;\Varid{a})\Rightarrow \Conid{Functor}\;(\Conid{E}\;\Varid{a})\;\mathbf{where}}$\\
${\hskip2.00em\relax\Varid{fmap}\;\Varid{f}\;(\Conid{Val}\;\Varid{n})\mathrel{=}\Conid{Val}\;\Varid{n}}$\\
${\hskip2.00em\relax\Varid{fmap}\;\Varid{f}\;(\Conid{Add}\;\Varid{e1}\;\Varid{e2})\mathrel{=}\Conid{Add}\;(\Varid{f}\;\Varid{e1})\;(\Varid{f}\;\Varid{e2})}$\\
${}$\\
${\mbox{\qquad-{}-  combine :: (Num a) => (E a) a -> a}}$\\
${\hskip1.00em\relax\Varid{combine}\mathbin{::}(\Conid{Num}\;\Varid{a})\Rightarrow \Conid{Algebra}\;(\Conid{E}\;\Varid{a})\;\Varid{a}}$\\
${\hskip1.00em\relax\Varid{combine}\;(\Conid{Val}\;\Varid{n})\mathrel{=}\Varid{n}}$\\
${\hskip1.00em\relax\Varid{combine}\;(\Conid{Add}\;\Varid{n1}\;\Varid{n2})\mathrel{=}(\Varid{n1}\mathbin{+}\Varid{n2})}$\\
${}$\\
${\hskip1.00em\relax\mathbf{instance}\;(\Conid{Num}\;\Varid{a})\Rightarrow (\Conid{AlgebraConstructor}\;(\Conid{E}\;\Varid{a})\;\Varid{a})\;\mathbf{where}}$\\
${\hskip1.00em\relax\hskip2.00em\relax\Varid{\phi}\mathrel{=}\Varid{combine}}$\\
${}$\\
${\hskip1.00em\relax\Varid{evalExpr}\mathbin{::}(\Conid{Num}\;\Varid{a})\Rightarrow (\Conid{Expr}\;\Varid{a})\to \Varid{a}}$\\
${\hskip1.00em\relax\Varid{evalExpr}\mathrel{=}\Varid{fold}\;\Varid{combine}}$
\end{tabbing}
We can use \ensuremath{\Varid{evalExprCata}} to get the same result with much less work:

\begin{tabbing}
\qquad\=\hspace{\lwidth}\=\hspace{\cwidth}\=\+\kill
${\hskip1.00em\relax\Varid{evalExprCata}\mathbin{::}(\Conid{Num}\;\Varid{a})\Rightarrow (\Conid{Expr}\;\Varid{a})\to \Varid{a}}$\\
${\hskip1.00em\relax\Varid{evalExprCata}\mathrel{=}\Varid{pCata}}$
\end{tabbing}
\ensuremath{\Varid{test}_1} is a sample expression for evaluation: $(3 + ((1 + 10) + 4))$

\begin{tabbing}
\qquad\=\hspace{\lwidth}\=\hspace{\cwidth}\=\+\kill
${\hskip1.00em\relax\Varid{test}_1\mathrel{=}(\Conid{In}\;(\Conid{Add}\;(\Conid{In}\;(\Conid{Val}\;\mathrm{3}))}$\\
${\hskip1.00em\relax\phantom{\Varid{test}_1\mathrel{=}(\Conid{In}\;(\mbox{}}(\Conid{In}\;(\Conid{Add}\;(\Conid{In}\;(\Conid{Add}\;(\Conid{In}\;(\Conid{Val}\;\mathrm{1}))}$\\
${\hskip1.00em\relax\phantom{\Varid{test}_1\mathrel{=}(\Conid{In}\;(\mbox{}}\phantom{(\Conid{In}\;(\Conid{Add}\;(\Conid{In}\;(\mbox{}}(\Conid{In}\;(\Conid{Val}\;\mathrm{10}))))}$\\
${\hskip1.00em\relax\phantom{\Varid{test}_1\mathrel{=}(\Conid{In}\;(\mbox{}}\phantom{(\Conid{In}\;(\mbox{}}(\Conid{In}\;(\Conid{Val}\;\mathrm{4}))))))}$
\end{tabbing}
To combine with Duponcheel, we would like to compose our semantic
evaluation functions into a single function.  For this, we will use a
\ensuremath{\Conid{Sum}} type defined exactly as in Duponcheel:

\begin{tabbing}
\qquad\=\hspace{\lwidth}\=\hspace{\cwidth}\=\+\kill
${\hskip1.00em\relax\mathbf{newtype}\;\Conid{Sum}\;\Varid{f}\;\Varid{g}\;\Varid{x}\mathrel{=}\Conid{S}\;(\Conid{Either}\;(\Varid{f}\;\Varid{x})\;(\Varid{g}\;\Varid{x}))}$\\
${}$\\
${\hskip1.00em\relax\Varid{unS}\;(\Conid{S}\;\Varid{x})\mathrel{=}\Varid{x}}$
\end{tabbing}
and show that \ensuremath{\Conid{Sum}} types are members of the \ensuremath{\Conid{Functor}} and
\ensuremath{\Conid{AlgebraConstructor}} classes so we can use \ensuremath{\Varid{pCata}} as our evaluation
function.

\begin{tabbing}
\qquad\=\hspace{\lwidth}\=\hspace{\cwidth}\=\+\kill
${\hskip1.00em\relax\mathbf{instance}\;(\Conid{Functor}\;\Varid{f},\Conid{Functor}\;\Varid{g})\Rightarrow \Conid{Functor}\;(\Conid{Sum}\;\Varid{f}\;\Varid{g})}$\\
${\hskip1.00em\relax\hskip2.00em\relax\mathbf{where}\;\Varid{fmap}\;\Varid{h}\;(\Conid{S}\;(\Conid{Left}\;\Varid{x}))\mathrel{=}\Conid{S}\;(\Conid{Left}\mathbin{\$}\Varid{fmap}\;\Varid{h}\;\Varid{x})}$\\
${\hskip1.00em\relax\hskip2.00em\relax\phantom{\mathbf{where}\;\mbox{}}\Varid{fmap}\;\Varid{h}\;(\Conid{S}\;(\Conid{Right}\;\Varid{x}))\mathrel{=}\Conid{S}\;(\Conid{Right}\mathbin{\$}\Varid{fmap}\;\Varid{h}\;\Varid{x})}$
\end{tabbing}
Here, using catamorphism instead of just fold, is a big win.
the definition:

\begin{tabbing}
\qquad\=\hspace{\lwidth}\=\hspace{\cwidth}\=\+\kill
${\hskip1.00em\relax\Varid{\phi}\mathrel{=}\Varid{either}\;\Varid{\phi}\;\Varid{\phi}\mathbin{\circ}\Varid{unS}}$
\end{tabbing}
uses \ensuremath{\Varid{\phi}} to refer to 3 different implementations \ensuremath{\Varid{\phi}}.  We define
\ensuremath{\Varid{\phi}} for the \ensuremath{\Conid{Sum}} type, using the \ensuremath{\Varid{\phi}} of each of the types we are
\ensuremath{\Conid{Sum}}'ing.  This is the result of using the \ensuremath{\Conid{Algebra}} type class
synonym rather than defining a separately named \ensuremath{\Varid{\phi}} for each term.

\begin{tabbing}
\qquad\=\hspace{\lwidth}\=\hspace{\cwidth}\=\+\kill
${\hskip1.00em\relax\mathbf{instance}\;(\Conid{AlgebraConstructor}\;\Varid{f}\;\Varid{a},\Conid{AlgebraConstructor}\;\Varid{g}\;\Varid{a})\Rightarrow }$\\
${\hskip1.00em\relax\hskip2.00em\relax\Conid{AlgebraConstructor}\;(\Conid{Sum}\;\Varid{f}\;\Varid{g})\;\Varid{a}}$\\
${\mbox{\qquad-{}-      phi :: Algebra fa == f a -> a}}$\\
${\hskip5.00em\relax\mathbf{where}\;\Varid{\phi}\mathrel{=}\Varid{either}\;\Varid{\phi}\;\Varid{\phi}\mathbin{\circ}\Varid{unS}}$
\end{tabbing}
As an example we extend Expressions defined above to include
multiplication.  I use this simple extension for brevity.  For
\emph{any} extension, all we need to do is:

\begin{enumerate}
  \parskip=0pt\itemsep=0pt
\item Define a new type for the extension
\item Show the extension is belongs to Functor and AlgebraConstructor
\item Define a new type, using Sum, to combine the original type with
  the extension type.
\end{enumerate}

And, we have an evaluator for the more complex type, without modifying
the base type and without defining any interaction between the two
types.

\begin{tabbing}
\qquad\=\hspace{\lwidth}\=\hspace{\cwidth}\=\+\kill
${\hskip1.00em\relax\mathbf{data}\;(\Conid{Num}\;\Varid{numType})\Rightarrow \Conid{M}\;\Varid{numType}\;\Varid{e}\mathrel{=}\Conid{Times}\;\Varid{e}\;\Varid{e}}$\\
${}$\\
${\hskip1.00em\relax\mathbf{instance}\;(\Conid{Num}\;\Varid{a})\Rightarrow \Conid{Functor}\;(\Conid{M}\;\Varid{a})\;\mathbf{where}}$\\
${\hskip1.00em\relax\hskip2.00em\relax\Varid{fmap}\;\Varid{f}\;(\Conid{Times}\;\Varid{e1}\;\Varid{e2})\mathrel{=}\Conid{Times}\;(\Varid{f}\;\Varid{e1})\;(\Varid{f}\;\Varid{e2})}$\\
${}$\\
${\hskip1.00em\relax\mathbf{instance}\;(\Conid{Num}\;\Varid{a})\Rightarrow (\Conid{AlgebraConstructor}\;(\Conid{M}\;\Varid{a})\;\Varid{a})\;\mathbf{where}}$\\
${\hskip1.00em\relax\hskip2.00em\relax\Varid{\phi}\;(\Conid{Times}\;\Varid{x}\;\Varid{y})\mathrel{=}\Varid{x}\mathbin{*}\Varid{y}}$\\
${}$\\
${\hskip1.00em\relax\mathbf{data}\;\Conid{BigE}\;\Varid{n}\;\Varid{a}\mathrel{=}\Conid{BE}\;(\Conid{Sum}\;(\Conid{E}\;\Varid{n})\;(\Conid{M}\;\Varid{n})\;\Varid{a})}$\\
${}$\\
${\hskip1.00em\relax\mathbf{type}\;\Conid{BigExpr}\;\Varid{numType}\mathrel{=}\Conid{Fix}\;(\Conid{BigE}\;\Varid{numType})}$
\end{tabbing}
Just a bit of boiler plate stuff.  Since \texttt{Haskell} requires
that the new type, \ensuremath{\Conid{BigE}}, use a data constructor (named, \ensuremath{\Conid{BE}}, here);
we have to show that this type is a member of \ensuremath{\Conid{Functor}} and
\ensuremath{\Conid{AlgebraConstructor}} by simply ripping off this data constructor.

\begin{tabbing}
\qquad\=\hspace{\lwidth}\=\hspace{\cwidth}\=\+\kill
${\hskip1.00em\relax\mathbf{instance}\;(\Conid{Num}\;\Varid{a})\Rightarrow \Conid{Functor}\;(\Conid{BigE}\;\Varid{a})\;\mathbf{where}}$\\
${\hskip1.00em\relax\hskip2.00em\relax\Varid{fmap}\;\Varid{f}\;(\Conid{BE}\;\Varid{x})\mathrel{=}\Conid{BE}\;(\Varid{fmap}\;\Varid{f}\;\Varid{x})}$\\
${}$\\
${\hskip1.00em\relax\mathbf{instance}\;(\Conid{Num}\;\Varid{a})\Rightarrow (\Conid{AlgebraConstructor}\;(\Conid{BigE}\;\Varid{a})\;\Varid{a})\;\mathbf{where}}$\\
${\hskip1.00em\relax\hskip2.00em\relax\Varid{\phi}\;(\Conid{BE}\;\Varid{x})\mathrel{=}\Varid{\phi}\;\Varid{x}}$\\
${}$\\
${\hskip1.00em\relax\Varid{evalBigExpr}\mathbin{::}(\Conid{Num}\;\Varid{a})\Rightarrow (\Conid{BigExpr}\;\Varid{a})\to \Varid{a}}$\\
${\hskip1.00em\relax\Varid{evalBigExpr}\mathrel{=}\Varid{pCata}}$
\end{tabbing}
We can do the same thing with fold, but is a bit more clumsy.
Now we must explicitly define the function to fold'ed over the sum.

The class \ensuremath{\Conid{AlgebraConstructor}} (and then catamorphism) are able to
eliminate this step, since this function is already known,
with the name \ensuremath{\Varid{\phi}}.

\begin{tabbing}
\qquad\=\hspace{\lwidth}\=\hspace{\cwidth}\=\+\kill
${\hskip1.00em\relax\Varid{unBE}\;(\Conid{BE}\;\Varid{x})\mathrel{=}\Varid{x}}$\\
${}$\\
${\hskip1.00em\relax\Varid{combineM}\mathbin{::}(\Conid{Num}\;\Varid{a})\Rightarrow (\Conid{M}\;\Varid{a})\;\Varid{a}\to \Varid{a}}$\\
${\hskip1.00em\relax\Varid{combineM}\;(\Conid{Times}\;\Varid{x}\;\Varid{y})\mathrel{=}\Varid{x}\mathbin{*}\Varid{y}}$\\
${}$\\
${\hskip1.00em\relax\Varid{summedCombine}\mathbin{::}(\Conid{Num}\;\Varid{a})\Rightarrow (\Conid{BigE}\;\Varid{a})\;\Varid{a}\to \Varid{a}}$\\
${\hskip1.00em\relax\Varid{summedCombine}\mathrel{=}\Varid{either}\;\Varid{combine}\;\Varid{combineM}\mathbin{\circ}\Varid{unS}\mathbin{\circ}\Varid{unBE}}$\\
${}$\\
${\hskip1.00em\relax\Varid{evalBigExprFold}\mathbin{::}(\Conid{Num}\;\Varid{a})\Rightarrow (\Conid{BigExpr}\;\Varid{a})\to \Varid{a}}$\\
${\hskip1.00em\relax\Varid{evalBigExprFold}\mathrel{=}\Varid{fold}\;\Varid{summedCombine}}$
\end{tabbing}
\section{Usage}

This file is a template for transforming literate script into \LaTeX
and is not actually a \texttt{Haskell} interpreter implementation.
Each section in this file is a separate module that can be loaded
individually for experimentation.


Note that the interpreters have been developed under GHC and some
require turning on the Glasgow Extensions.  Your mileage may vary if
you're using HUGS.

To build a \LaTeX document from the interpreter files, use:

\begin{alltt}
   lhs2TeX --math Hutton.lhs > Hutton.tex
\end{alltt}

and run \LaTeX on the result.  The individual interpreters cannot be
transformed to \LaTeX directly.

\end{document}